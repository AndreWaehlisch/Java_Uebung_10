\documentclass{scrartcl}

\usepackage[utf8]{inputenc} %UTF8 ohne BOM!
\usepackage[T1]{fontenc}
\usepackage[german]{babel}

\usepackage{amssymb,amsmath,booktabs,multirow,csquotes}

\usepackage{listings}
\lstset{numbers=left, numberstyle=\normalsize, stepnumber=1, frame=single, breaklines=true, mathescape=true, literate=%
    {Ö}{{\"O}}1
    {Ä}{{\"A}}1
    {Ü}{{\"U}}1
    {ß}{{\ss}}1
    {ü}{{\"u}}1
    {ä}{{\"a}}1
    {ö}{{\"o}}1
    {~}{{\textasciitilde}}1
}

\usepackage{hyperref}

\begin{document}

\title{10. Übung Praktisches Programmieren und Rechneraufbau (Java)}

\date{Abgabe: \today}

\author{
\large
\textsc{André Wählisch}\\[2mm]
\normalsize \href{mailto:andre.waehlisch@physik.tu-berlin.de}{andre.waehlisch@physik.tu-berlin.de} \\
\normalsize Matrikelnummer 325428 
}

\maketitle

\section*{Aufgabe 1}

Das Diagramm der Vererbungshierarchie befindet sich in der Datei \enquote{diagram.png}. Der entsprechende Quellcode befindet sich im Unterordner \enquote{code}.

\clearpage

\section*{Aufgabe 2}

%\begin{landscape}
%\thispagestyle{empty}
\begin{center}
\begin{tabular}{ccccccc} % 33 Zeilen
	\toprule
	Zeile	& z	& feld$\left[0\right]$	& feld$\left[1\right]$	& s				& i	& Bemerkung\\
	\midrule
	5		& -	& -						& -						& -				& -	& Anfang main-Methode\\
	6		& 	& 0						& 0						& 				& 	& Automatische Initialisierung mit Nullen\\
	7		&	&						&						& undef			&	& \\
	8		&	&						&						&				& 0	& \\
	9		&	&						&						&				&	& \\ % try{
	\midrule
	2(10)	& 1	&						&						&				&	& \\
	3(10)	&	&						&						&				&	& Teilen durch 0 $\Rightarrow$ ArithmeticException\\
	4(10)	& -	&						&						&				&	& \\
	\midrule
	16		&	&						&						&				&	& \\
	17		&	& 5						&						&				&	& \\
	18		&	&						&						&				&	& \\
	19		&	&						&						&				& 1	& \\
	8		&	&						&						&				&	& Schleifen-Bedingung okay\\
	9		&	&						&						&				&	& \\ % try{
	\midrule
	2(10)	& 2	&						&						&				&	& \\
	3(10)	&	&						&						&				&	& return-Wert: $\frac{2*2}{2-1}=4$\\
	4(10)	& -	&						&						&				&	& \\
	\midrule
	10		& 	&						& 4						&				&	& \\
	11		&	&						&						& \enquote{Halbzeit}	&	& \\
	12		&	&						&						&				&	& \\
	19		&	&						&						&				& 2	& \\
	8		&	&						&						&				&	& Schleifen-Bedingung okay\\
	9		&	&						&						&				&	& \\ % try{
	\midrule
	2(10)	& 3	&						&						&				&	& \\
	3(10)	&	&						&						&				&	& return-Wert: $\frac{2*3}{3-1}=3$\\
	4(10)	& -	&						&						&				&	& \multirow{2}{4cm}{ArrayIndexOutOfBoundsException da \emph{feld$\left[2\right]$} nicht existiert}\\
			&	&						&						&				&	& \\ % leere Zeile
	\midrule
	13		&	&						&						&				&	& \\
	14		&	&						&						& \enquote{Ende}		&	& \\
	15		&	&						&						&				&	& \\
	19		&	&						&						&				& 3	& \\
	8		&	&						&						&				&	& Schleifenbedingung nicht mehr erfüllt\\
	19		&	&						&						&				& -	& \\
	20		&	& -						& -						& -				& 	& Ende main-Methode\\
	\bottomrule
\end{tabular}
\end{center}
%\end{landscape}

\end{document}